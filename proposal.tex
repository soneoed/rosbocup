\documentclass{llncs}

% -------------------------------------------- Additional Packages not defined in Paper package ---------------------- %
\usepackage{url}							% ------- formats urls nicely
\usepackage{algorithmic}
\usepackage[sort,compress]{cite}

% ---------------------- Font packages
\usepackage[T1]{fontenc}
\usepackage[latin1]{inputenc}					% ------- Include accented characters
\usepackage{color}							% ------- Font Colours eg. /textcolor{red}{text that will be red}
\usepackage{soul}

% ---------------------- Math Packages
\usepackage{amsmath}
\usepackage{amsfonts}

% ---------------------- Graphics
\usepackage{graphicx} %%graphics and normal LaTeX [dvips]
\usepackage[scriptsize]{subfigure}

% ---------------------- Hyper Referencing
%\usepackage{hyperref}
%\hypersetup{colorlinks=true,	% set this to false when printing
%linkcolor = blue,	 		%Color for normal internal links.
%anchorcolor = black, 		%Color for anchor text. 
%citecolor = blue, 			%Color for bibligraphical citations in text. 
%filecolor = magenta, 			%Color for URLs which open local �les. 
%menucolor = blue,			%Color for Acrobat menu items. 
%pagecolor = blue, 			%Color for links to other pages. 
%urlcolor = blue, 			%Color for linked URLs. 
%}

% ---------------------- Additional Tools
\usepackage{url}			% formats urls nicely
\usepackage{unitsdef}		% contains definitions for physical units



% ------------------------------------------------------------------ Begin Document -----------------------------------------------------------------------------------------------

\title{ROSboCup: A ROS--based Framework for RoboCup Soccer}

\author{Robert King\inst{1} \and Shashank Bhatia\inst{2} \and Jason Kulk\inst{2}}

\institute{
School of Mathematical and Physical Sciences\\ 
University of Newcastle, Australia\\
\email{robert.king@newcastle.edu.au}
\and
School of Electrical Engineering and Computer Science\\ 
University of Newcastle, Australia\\
\email{\{shashank.bhatia, jason.kulk\}@newcastle.edu.au}
}

\begin{document}

\maketitle
\thispagestyle{empty}
\pagestyle{empty}

% I always forget how to write an abstract:
% What is the general topic and why it is important
% What is the specific problem? What methods. The results
% Wider significance.

\begin{abstract}
The development of software systems to compete in RoboCup's soccer leagues is a large and complicated task. The software framework on which the system is based places restrictions on the software in terms of supporting multiple platforms and leagues, and on its modularity. 

There are many existing frameworks in the RoboCup domain, however, they are not widely used outside of the authoring team, and are not shared across different robot platforms or leagues. Furthermore, the frameworks have little application outside of RoboCup.

In this proposal, we suggest that a community owned open--source framework for RoboCup's soccer leagues be developed based on Willow Garage's Robot Operating System (ROS).

\end{abstract}

% ------------------------------------------------------------------ Main Text -----------------------------------------------------------------------------------------------
A significant part of the time spent while preparing an entry into a RoboCup
soccer league goes into the development of the underlying software framework.
Essentially, a framework provides an abstraction layer to allow communication
between hardware and software entities. For the robust functioning of the entire
robocup logic, the framework also caters for all
requirements of the underlying platform. Further, it provides a set of tools such as serialisation and debugging which
come handy while the teams concentrate on developments of the
robocup logic.

There are numerous examples of software frameworks used in the RoboCup soccer
arena \cite{Mcgill2010,Petters2007,Rofer2010b,Kulk2011c}. For several
reasons, existing frameworks are not widely adopted by the community. Most of
these frameworks, are designed while being focused on a particular
robot platform and league. While this makes the design easy, however renders
the reusability of the framework by other teams (having different robot
platforms) impossible.

From the implementation point of view, sometimes frameworks enforce the teams
to adapt to non--standard solutions to
standard problems. As an instance, \cite{Rofer2010b} offers customised
implementations of vectors and serialisation, which are rather more efficienty
available in the standard C++ libraries like STL and boost. This presents a
problem to the wide--spread adoption because of the reluctance of the teams to
learn these custom libraries which would not be used outside of RoboCup.

In terms of the availability of source code, most existing
software frameworks do not release the barebones separately. In the senario
of most of the robocup competitions, these frameworks are rather released as
part of the team's code. Thereby leaving the new teams with additional work of
segregating the robocup logic from the underlying framework. This is especially
a hinderance to the new teams, who wish to provide different robocup logic.

To solve the above discussed problems in the existing frameworks, Willow
Garage's Robot Operating System (ROS) comes as an inspiration. ROS is a
component--based software framework for robotics \cite{Quigley2009}. It has an
active community and is widely used in research projects and in industry.
Consequently, the use of ROS by RoboCup teams is likely to increase
significantly in the coming years. However, adaption of ROS alone does not
standardise the software system enough to guarantee the interoperability of the
software modules required by a RoboCup soccer team.

Therefore, we propose the use of the ROS framework to construct a open--source
framework to meet the needs of the RoboCup soccer community. In particular, support for
the NAO, the DARWIN-OP, and the 2D and 3D simulation leagues will be added.
Furthermore, standard interfaces will be implemented for the common software
modules; vision, localisation, behaviour and motion.

The new framework will benefit the RoboCup community by not only allowing the
effective sharing of code between teams of the same league, but also between
leagues. It will provide a barebone platform independent starting point for
anyone trying to get involved into the Robocup competitions, thereby
encouraging new teams into the competition. 


% ------------------------------------------------------------------ References -----------------------------------------------------------------------------------------------
\bibliographystyle{IEEEtran}
\bibliography{rosbocup}

\end{document}
